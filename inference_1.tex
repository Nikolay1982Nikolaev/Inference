% Options for packages loaded elsewhere
\PassOptionsToPackage{unicode}{hyperref}
\PassOptionsToPackage{hyphens}{url}
%
\documentclass[
]{article}
\usepackage{amsmath,amssymb}
\usepackage{iftex}
\ifPDFTeX
  \usepackage[T1]{fontenc}
  \usepackage[utf8]{inputenc}
  \usepackage{textcomp} % provide euro and other symbols
\else % if luatex or xetex
  \usepackage{unicode-math} % this also loads fontspec
  \defaultfontfeatures{Scale=MatchLowercase}
  \defaultfontfeatures[\rmfamily]{Ligatures=TeX,Scale=1}
\fi
\usepackage{lmodern}
\ifPDFTeX\else
  % xetex/luatex font selection
\fi
% Use upquote if available, for straight quotes in verbatim environments
\IfFileExists{upquote.sty}{\usepackage{upquote}}{}
\IfFileExists{microtype.sty}{% use microtype if available
  \usepackage[]{microtype}
  \UseMicrotypeSet[protrusion]{basicmath} % disable protrusion for tt fonts
}{}
\makeatletter
\@ifundefined{KOMAClassName}{% if non-KOMA class
  \IfFileExists{parskip.sty}{%
    \usepackage{parskip}
  }{% else
    \setlength{\parindent}{0pt}
    \setlength{\parskip}{6pt plus 2pt minus 1pt}}
}{% if KOMA class
  \KOMAoptions{parskip=half}}
\makeatother
\usepackage{xcolor}
\usepackage[margin=1in]{geometry}
\usepackage{color}
\usepackage{fancyvrb}
\newcommand{\VerbBar}{|}
\newcommand{\VERB}{\Verb[commandchars=\\\{\}]}
\DefineVerbatimEnvironment{Highlighting}{Verbatim}{commandchars=\\\{\}}
% Add ',fontsize=\small' for more characters per line
\usepackage{framed}
\definecolor{shadecolor}{RGB}{248,248,248}
\newenvironment{Shaded}{\begin{snugshade}}{\end{snugshade}}
\newcommand{\AlertTok}[1]{\textcolor[rgb]{0.94,0.16,0.16}{#1}}
\newcommand{\AnnotationTok}[1]{\textcolor[rgb]{0.56,0.35,0.01}{\textbf{\textit{#1}}}}
\newcommand{\AttributeTok}[1]{\textcolor[rgb]{0.13,0.29,0.53}{#1}}
\newcommand{\BaseNTok}[1]{\textcolor[rgb]{0.00,0.00,0.81}{#1}}
\newcommand{\BuiltInTok}[1]{#1}
\newcommand{\CharTok}[1]{\textcolor[rgb]{0.31,0.60,0.02}{#1}}
\newcommand{\CommentTok}[1]{\textcolor[rgb]{0.56,0.35,0.01}{\textit{#1}}}
\newcommand{\CommentVarTok}[1]{\textcolor[rgb]{0.56,0.35,0.01}{\textbf{\textit{#1}}}}
\newcommand{\ConstantTok}[1]{\textcolor[rgb]{0.56,0.35,0.01}{#1}}
\newcommand{\ControlFlowTok}[1]{\textcolor[rgb]{0.13,0.29,0.53}{\textbf{#1}}}
\newcommand{\DataTypeTok}[1]{\textcolor[rgb]{0.13,0.29,0.53}{#1}}
\newcommand{\DecValTok}[1]{\textcolor[rgb]{0.00,0.00,0.81}{#1}}
\newcommand{\DocumentationTok}[1]{\textcolor[rgb]{0.56,0.35,0.01}{\textbf{\textit{#1}}}}
\newcommand{\ErrorTok}[1]{\textcolor[rgb]{0.64,0.00,0.00}{\textbf{#1}}}
\newcommand{\ExtensionTok}[1]{#1}
\newcommand{\FloatTok}[1]{\textcolor[rgb]{0.00,0.00,0.81}{#1}}
\newcommand{\FunctionTok}[1]{\textcolor[rgb]{0.13,0.29,0.53}{\textbf{#1}}}
\newcommand{\ImportTok}[1]{#1}
\newcommand{\InformationTok}[1]{\textcolor[rgb]{0.56,0.35,0.01}{\textbf{\textit{#1}}}}
\newcommand{\KeywordTok}[1]{\textcolor[rgb]{0.13,0.29,0.53}{\textbf{#1}}}
\newcommand{\NormalTok}[1]{#1}
\newcommand{\OperatorTok}[1]{\textcolor[rgb]{0.81,0.36,0.00}{\textbf{#1}}}
\newcommand{\OtherTok}[1]{\textcolor[rgb]{0.56,0.35,0.01}{#1}}
\newcommand{\PreprocessorTok}[1]{\textcolor[rgb]{0.56,0.35,0.01}{\textit{#1}}}
\newcommand{\RegionMarkerTok}[1]{#1}
\newcommand{\SpecialCharTok}[1]{\textcolor[rgb]{0.81,0.36,0.00}{\textbf{#1}}}
\newcommand{\SpecialStringTok}[1]{\textcolor[rgb]{0.31,0.60,0.02}{#1}}
\newcommand{\StringTok}[1]{\textcolor[rgb]{0.31,0.60,0.02}{#1}}
\newcommand{\VariableTok}[1]{\textcolor[rgb]{0.00,0.00,0.00}{#1}}
\newcommand{\VerbatimStringTok}[1]{\textcolor[rgb]{0.31,0.60,0.02}{#1}}
\newcommand{\WarningTok}[1]{\textcolor[rgb]{0.56,0.35,0.01}{\textbf{\textit{#1}}}}
\usepackage{longtable,booktabs,array}
\usepackage{calc} % for calculating minipage widths
% Correct order of tables after \paragraph or \subparagraph
\usepackage{etoolbox}
\makeatletter
\patchcmd\longtable{\par}{\if@noskipsec\mbox{}\fi\par}{}{}
\makeatother
% Allow footnotes in longtable head/foot
\IfFileExists{footnotehyper.sty}{\usepackage{footnotehyper}}{\usepackage{footnote}}
\makesavenoteenv{longtable}
\usepackage{graphicx}
\makeatletter
\def\maxwidth{\ifdim\Gin@nat@width>\linewidth\linewidth\else\Gin@nat@width\fi}
\def\maxheight{\ifdim\Gin@nat@height>\textheight\textheight\else\Gin@nat@height\fi}
\makeatother
% Scale images if necessary, so that they will not overflow the page
% margins by default, and it is still possible to overwrite the defaults
% using explicit options in \includegraphics[width, height, ...]{}
\setkeys{Gin}{width=\maxwidth,height=\maxheight,keepaspectratio}
% Set default figure placement to htbp
\makeatletter
\def\fps@figure{htbp}
\makeatother
\setlength{\emergencystretch}{3em} % prevent overfull lines
\providecommand{\tightlist}{%
  \setlength{\itemsep}{0pt}\setlength{\parskip}{0pt}}
\setcounter{secnumdepth}{-\maxdimen} % remove section numbering
\ifLuaTeX
  \usepackage{selnolig}  % disable illegal ligatures
\fi
\IfFileExists{bookmark.sty}{\usepackage{bookmark}}{\usepackage{hyperref}}
\IfFileExists{xurl.sty}{\usepackage{xurl}}{} % add URL line breaks if available
\urlstyle{same}
\hypersetup{
  pdftitle={inference\_1},
  pdfauthor={NikolayNikolaev},
  hidelinks,
  pdfcreator={LaTeX via pandoc}}

\title{inference\_1}
\author{NikolayNikolaev}
\date{2023-05-18}

\begin{document}
\maketitle

\hypertarget{concept-1.2-multiplication-rule}{%
\subsection{Concept 1.2 (Multiplication
Rule):}\label{concept-1.2-multiplication-rule}}

\begin{Shaded}
\begin{Highlighting}[]
\CommentTok{\#define the vectors for size, topping and order}
\NormalTok{size }\OtherTok{=} \FunctionTok{c}\NormalTok{(}\StringTok{"S"}\NormalTok{, }\StringTok{"M"}\NormalTok{, }\StringTok{"L"}\NormalTok{)}
\NormalTok{topping }\OtherTok{=} \FunctionTok{c}\NormalTok{(}\StringTok{"pepperoni"}\NormalTok{, }\StringTok{"sausage"}\NormalTok{, }\StringTok{"meatball"}\NormalTok{, }\StringTok{"extra cheese"}\NormalTok{)}
\NormalTok{order }\OtherTok{=} \FunctionTok{c}\NormalTok{(}\StringTok{"deliver"}\NormalTok{, }\StringTok{"pick{-}up"}\NormalTok{)}

\CommentTok{\#keep track of the pizzas}
\NormalTok{pizzas }\OtherTok{=} \FunctionTok{character}\NormalTok{(}\DecValTok{0}\NormalTok{)}

\CommentTok{\#iterate over each value for each variable}
\ControlFlowTok{for}\NormalTok{(i }\ControlFlowTok{in} \DecValTok{1}\SpecialCharTok{:}\FunctionTok{length}\NormalTok{(size))\{}
  \ControlFlowTok{for}\NormalTok{(j }\ControlFlowTok{in} \DecValTok{1}\SpecialCharTok{:}\FunctionTok{length}\NormalTok{(topping))\{}
    \ControlFlowTok{for}\NormalTok{(k }\ControlFlowTok{in} \DecValTok{1}\SpecialCharTok{:}\FunctionTok{length}\NormalTok{(order))\{}
      
      \CommentTok{\#create a pizza}
\NormalTok{      pizzas }\OtherTok{=} \FunctionTok{rbind}\NormalTok{(pizzas, }\FunctionTok{c}\NormalTok{(size[i], topping[j], order[k]))}
\NormalTok{    \}}
\NormalTok{  \}}
\NormalTok{\}}

\CommentTok{\#print out the pizzas; should have 24}
\NormalTok{pizzas}
\end{Highlighting}
\end{Shaded}

\begin{verbatim}
##       [,1] [,2]           [,3]     
##  [1,] "S"  "pepperoni"    "deliver"
##  [2,] "S"  "pepperoni"    "pick-up"
##  [3,] "S"  "sausage"      "deliver"
##  [4,] "S"  "sausage"      "pick-up"
##  [5,] "S"  "meatball"     "deliver"
##  [6,] "S"  "meatball"     "pick-up"
##  [7,] "S"  "extra cheese" "deliver"
##  [8,] "S"  "extra cheese" "pick-up"
##  [9,] "M"  "pepperoni"    "deliver"
## [10,] "M"  "pepperoni"    "pick-up"
## [11,] "M"  "sausage"      "deliver"
## [12,] "M"  "sausage"      "pick-up"
## [13,] "M"  "meatball"     "deliver"
## [14,] "M"  "meatball"     "pick-up"
## [15,] "M"  "extra cheese" "deliver"
## [16,] "M"  "extra cheese" "pick-up"
## [17,] "L"  "pepperoni"    "deliver"
## [18,] "L"  "pepperoni"    "pick-up"
## [19,] "L"  "sausage"      "deliver"
## [20,] "L"  "sausage"      "pick-up"
## [21,] "L"  "meatball"     "deliver"
## [22,] "L"  "meatball"     "pick-up"
## [23,] "L"  "extra cheese" "deliver"
## [24,] "L"  "extra cheese" "pick-up"
\end{verbatim}

\begin{Shaded}
\begin{Highlighting}[]
\CommentTok{\#count total number of large sausages; should get 2}
\CommentTok{\#   we divide by 3 because rows are length 3, and we want}
\CommentTok{\#   to convert back to number of rows (i.e., number of pizzas)}
\FunctionTok{length}\NormalTok{(pizzas[pizzas[ ,}\DecValTok{1}\NormalTok{] }\SpecialCharTok{==} \StringTok{"L"} \SpecialCharTok{\&}\NormalTok{ pizzas[ ,}\DecValTok{2}\NormalTok{] }\SpecialCharTok{==} \StringTok{"sausage"}\NormalTok{])}\SpecialCharTok{/}\DecValTok{3}
\end{Highlighting}
\end{Shaded}

\begin{verbatim}
## [1] 2
\end{verbatim}

\hypertarget{concept-1.3-factorial}{%
\subsection{Concept 1.3 (Factorial):}\label{concept-1.3-factorial}}

\begin{itemize}
\tightlist
\item
  permutation: A ,B, C = 3! = 3x2x1 = 6
\item
  {[}ABC, ACB, BAC, BCA, CAB, CBA{]}
\end{itemize}

\begin{Shaded}
\begin{Highlighting}[]
\FunctionTok{library}\NormalTok{(combinat)}
\end{Highlighting}
\end{Shaded}

\begin{verbatim}
## 
## Attaching package: 'combinat'
\end{verbatim}

\begin{verbatim}
## The following object is masked from 'package:utils':
## 
##     combn
\end{verbatim}

\begin{Shaded}
\begin{Highlighting}[]
\CommentTok{\#generate all of the possible permutations}
\NormalTok{perms }\OtherTok{=}\NormalTok{ combinat}\SpecialCharTok{::}\FunctionTok{permn}\NormalTok{(}\FunctionTok{c}\NormalTok{(}\StringTok{"A"}\NormalTok{, }\StringTok{"B"}\NormalTok{, }\StringTok{"C"}\NormalTok{, }\StringTok{"D"}\NormalTok{, }\StringTok{"E"}\NormalTok{, }\StringTok{"F"}\NormalTok{, }\StringTok{"G"}\NormalTok{))}

\CommentTok{\#look at the first few permutations}
\FunctionTok{head}\NormalTok{(perms)}
\end{Highlighting}
\end{Shaded}

\begin{verbatim}
## [[1]]
## [1] "A" "B" "C" "D" "E" "F" "G"
## 
## [[2]]
## [1] "A" "B" "C" "D" "E" "G" "F"
## 
## [[3]]
## [1] "A" "B" "C" "D" "G" "E" "F"
## 
## [[4]]
## [1] "A" "B" "C" "G" "D" "E" "F"
## 
## [[5]]
## [1] "A" "B" "G" "C" "D" "E" "F"
## 
## [[6]]
## [1] "A" "G" "B" "C" "D" "E" "F"
\end{verbatim}

\begin{Shaded}
\begin{Highlighting}[]
\CommentTok{\#should get factorial(7) = 5040}
\FunctionTok{length}\NormalTok{(perms)}
\end{Highlighting}
\end{Shaded}

\begin{verbatim}
## [1] 5040
\end{verbatim}

\hypertarget{concept-1.4-binomial-coefficient}{%
\subsection{Concept 1.4 (Binomial
Coefficient):}\label{concept-1.4-binomial-coefficient}}

\begin{itemize}
\tightlist
\item
  ``n chose x'' = \(\binom{n}{x} = \frac{n!}{(n-x)!.(x!)}\)
\item
  n=5, x = 3 =\textgreater{} \(\binom{5}{3} = \frac{5!}{(5-3)!.(3!)}\)
\item
  the binomial coefficient gives the number of ways that x objects can
  be chosen from a population of n objects
\end{itemize}

\begin{Shaded}
\begin{Highlighting}[]
\FunctionTok{library}\NormalTok{(gtools)}
\end{Highlighting}
\end{Shaded}

\begin{verbatim}
## Warning: package 'gtools' was built under R version 4.2.3
\end{verbatim}

\begin{Shaded}
\begin{Highlighting}[]
\CommentTok{\#generate the committees (people labeled 1 to 5)}
\NormalTok{committees }\OtherTok{=}\NormalTok{ gtools}\SpecialCharTok{::}\FunctionTok{combinations}\NormalTok{(}\AttributeTok{n =} \DecValTok{5}\NormalTok{, }\AttributeTok{r =} \DecValTok{3}\NormalTok{)}

\CommentTok{\#should get choose(5, 3) = factorial(5)/(factorial(3)*factorial(2)) = 10 committees}
\NormalTok{committees}
\end{Highlighting}
\end{Shaded}

\begin{verbatim}
##       [,1] [,2] [,3]
##  [1,]    1    2    3
##  [2,]    1    2    4
##  [3,]    1    2    5
##  [4,]    1    3    4
##  [5,]    1    3    5
##  [6,]    1    4    5
##  [7,]    2    3    4
##  [8,]    2    3    5
##  [9,]    2    4    5
## [10,]    3    4    5
\end{verbatim}

\begin{longtable}[]{@{}lll@{}}
\toprule\noalign{}
& Orfer Matters'' Order Doesn't Matter & \\
\midrule\noalign{}
\endhead
\bottomrule\noalign{}
\endlastfoot
with replacement & n\^{}k & binom\{n+k-1\}\{k\} \\
without replacement & \frac{n!}{(n-k)!} & \binom{n}{k} \\
\end{longtable}

\end{document}
